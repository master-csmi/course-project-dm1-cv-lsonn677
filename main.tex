% a mashup of hipstercv, friggeri and twenty cv
% https://www.latextemplates.com/template/twenty-seconds-resumecv
% https://www.latextemplates.com/template/friggeri-resume-cv

\documentclass[verylight]{simplehipstercv}
% available options are: darkhipster, lighthipster, pastel, allblack, grey, verylight, withoutsidebar
% withoutsidebar
\usepackage[utf8]{inputenc}
\usepackage[default]{raleway}
\usepackage[margin=1cm, a4paper]{geometry}
\usepackage{xcolor}

%------------------------------------------------------------------ Variablen

\newlength{\rightcolwidth}
\newlength{\leftcolwidth}
\setlength{\leftcolwidth}{0.23\textwidth}
\setlength{\rightcolwidth}{0.75\textwidth}

%------------------------------------------------------------------
\title{Mon CV}
\author{\LaTeX{} Lucas}
\date{Septembre 2025}

\pagestyle{empty}
\begin{document}


\thispagestyle{empty}
%-------------------------------------------------------------

\section*{Start}

\simpleheader{teal}{Lucas}{Sonntag}{étudiant en master CSMI}{white}



%------------------------------------------------

% this has to be here so the paracols starts..
\subsection*{}
\vspace{4em}

\setlength{\columnsep}{1.5cm}
\columnratio{0.23}[0.75]
\begin{paracol}{2}
\hbadness5000
%\backgroundcolor{c[1]}[rgb]{1,1,0.8} % cream yellow for column-1 %\backgroundcolor{g}[rgb]{0.8,1,1} % \backgroundcolor{l}[rgb]{0,0,0.7} % dark blue for left margin

\paracolbackgroundoptions

% 0.9,0.9,0.9 -- 0.8,0.8,0.8


\footnotesize
{\setasidefontcolour
\flushright
\begin{center}
    \roundpic{photo.png}
\end{center}

\bg{cvgreen}{white}{à propos de moi}\\[0.5em]

{\footnotesize
étudiant en master CSMI à l'université de Strasbourg,
je suis un jeune passionné de mathématiques sous toutes les formes qui soient, autant pures que appliquées. j'ai un intêret et un goût prononcé pour l'informatique et la recherche.
Je suis motivé et stimulé par la résolution de problèmes complexes.}
\bigskip

\bg{cvgreen}{white}{infos personnelles} \\[0.5em]
Lucas Sonntag   

nationalité: Français 

naissance: 02.03.04

\bigskip

\bg{cvgreen}{white}{Spécialisation} \\[0.5em]

analyse fonctionnelle
probabilité et statistiques
calcul scientifique


\bigskip



\bigskip

\bg{cvgreen}{white}{Domaine d'interet}\\[0.5em]

algorithmique
informatique
modélisation mathématique


\bigskip


\vspace{4em}

\infobubble{\faGithub}{cvgreen}{white}{lsonn667}

\phantom{turn the page}


}
%-----------------------------------------------------------
\switchcolumn

\small

\section*{Curriculum}
\begin{tabular}{r| p{0.5\textwidth} c}
    \cvevent{2019--2022}{Lycée général et technologique section abibac}{lycee alfred Kastler}{Guebwiller}{préparation du baccalauréat général ainsi que de l'abitur.}{logo_lycee.jpeg} \\
    \cvevent{2022--2025}{Licence de mathématique}{Ufr Mathématique et d'informatique}{université de strasbourg}{préparation d'une licence de mathématique fondamentale.}{logo_ufr.png} \\
\end{tabular}
\vspace{3em}



\begin{minipage}[t]{0.35\textwidth}
\section*{diplômes}
\begin{tabular}{r p{0.6\textwidth} c}
    \cvdegree{2022}{Baccalauréat général}{}{Guebwiller \color{headerblue}}{}{logo_lycee.jpeg} \\
    \cvdegree{2022}{Abiturzeugnisse}{Baccalauréat allemand}{Guebwiller \color{headerblue}}{}{logo_lycee.jpeg} \\
    \cvdegree{2025}{Licence de mathématique fondamentale}{}{Strasbourg \color{headerblue}}{}{logo_ufr.png}
\end{tabular}
\end{minipage}\hfill
\begin{minipage}[t]{0.3\textwidth}
\section*{Programmation}
\begin{tabular}{r @{\hspace{0.5em}}l}
     \bg{skilllabelcolour}{iconcolour}{html, css} &  \barrule{0.1}{0.5em}{cvgreen}\\
     \bg{skilllabelcolour}{iconcolour}{\LaTeX} & \barrule{0.5}{0.5em}{cvgreen} \\
     \bg{skilllabelcolour}{iconcolour}{python} & \barrule{0.5}{0.5em}{cvgreen} \\
     \bg{skilllabelcolour}{iconcolour}{C++} & \barrule{0.30}{0.5em}{cvgreen} \\
     \bg{skilllabelcolour}{iconcolour}{C} & \barrule{0.15}{0.5em}{cvgreen} \\
     \bg{skilllabelcolour}{iconcolour}{SQL} & \barrule{0.25}{0.5em}{cvgreen} \\
\end{tabular}
\end{minipage}

\section*{Projet académique}

\begin{tabular}{r| p{0.5\textwidth} c}
    \cvevent{2025}{projet de recherche sur la mesure de Haar}{}{Strasbourg \color{cvred}}{à l'occasion, de la L3 math fondamentale, j'ai pu experimenter un travail de recherche et de synthèse. J'ai du faire des recherches puis présenter et synthétiser mes résultats sur le sujet suivant: Les Mesures de Haar sur les groupes unimodulaires.}{logo_ufr.png} \\
\end{tabular}
\vspace{3em}



\section*{langues}
\begin{tabular}{l | ll}
\textbf{French} & C2 & {\phantom{x}\footnotesize langue maternelle} \\
\textbf{Allemand} & C1 & \pictofraction{\faCircle}{cvgreen}{4}{black!30}{1}{\tiny} \\
\textbf{Anglais} & B2 & \pictofraction{\faCircle}{cvgreen}{3}{black!30}{2}{\tiny} \\
\end{tabular}
\bigskip








\vfill{} % Whitespace before final footer

%----------------------------------------------------------------------------------------
%	FINAL FOOTER
%----------------------------------------------------------------------------------------
\setlength{\parindent}{0pt}
\begin{minipage}[t]{\rightcolwidth}
\begin{center}\fontfamily{\sfdefault}\selectfont \color{black!70}
{\small Sonntag Lucas \icon{\faAt}{cvgreen}{} \protect\url{lucas.sonntag68@gmail.com} \icon{\faMapMarker}{cvgreen}{} Alsace \icon{\faPhone}{cvgreen}{} +33 6.12.15.69.72
}
\end{center}
\end{minipage}

\end{paracol}

\end{document}
